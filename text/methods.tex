Coaxial HPGe spectra resulting from measurements of Co-60, Am-241, Cs-137,and Ba-133 check
sources were provided to students via a publicly-accessible dropbox folder. Using MAKE and a 
python script, this data was downloaded and loaded into NumPy arrays for
analysis. Peaks in the spectra corresponding to photopeaks of known energy were
then picked out and fitted with a gaussian distribution using SciPy's
curvefit tool. The channel location of the peaks resulting from this fitting process were used to 
create a calibration. To assure the utility of the calibration for common sources in the energy range of 60-600keV, the
661.7keV and 59.5keV lines from Cs-137 and Am-241 were used. For the sake of simplicity, 
this calibration was of the form: \[E_x = a*x + b\] (in which x denotes
the channel index).This calibration (a = 0.28058369, b = 0.9339521) was then applied to the dataset gathered from
a Ba-133 check source in order to determine the accuracy of the fit.
